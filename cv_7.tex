%%%%%%%%%%%%%%%%%%%%%%%%%%%%%%%%%%%%%%%%%
% "ModernCV" CV and Cover Letter
% LaTeX Template
% Version 1.11 (19/6/14)
%
% This template has been downloaded from:
% http://www.LaTeXTemplates.com
%
% Original author:
% Xavier Danaux (xdanaux@gmail.com)
%
% License:
% CC BY-NC-SA 3.0 (http://creativecommons.org/licenses/by-nc-sa/3.0/)
%
% Important note:
% This template requires the moderncv.cls and .sty files to be in the same 
% directory as this .tex file. These files provide the resume style and themes 
% used for structuring the document.
%
%%%%%%%%%%%%%%%%%%%%%%%%%%%%%%%%%%%%%%%%%

%----------------------------------------------------------------------------------------
%	PACKAGES AND OTHER DOCUMENT CONFIGURATIONS
%----------------------------------------------------------------------------------------

\documentclass[11pt,a4paper,sans]{moderncv} % Font sizes: 10, 11, or 12; paper sizes: a4paper, letterpaper, a5paper, legalpaper, executivepaper or landscape; font families: sans or roman

\moderncvstyle{banking} % CV theme - options include: 'casual' (default), 'classic', 'oldstyle' and 'banking'
\moderncvcolor{blue} % CV color - options include: 'blue' (default), 'orange', 'green', 'red', 'purple', 'grey' and 'black'

\usepackage{lipsum} % Used for inserting dummy 'Lorem ipsum' text into the template

\usepackage[scale=0.75]{geometry} % Reduce document margins
%\setlength{\hintscolumnwidth}{3cm} % Uncomment to change the width of the dates column
%\setlength{\makecvtitlenamewidth}{10cm} % For the 'classic' style, uncomment to adjust the width of the space allocated to your name

%----------------------------------------------------------------------------------------
%	NAME AND CONTACT INFORMATION SECTION
%----------------------------------------------------------------------------------------

\firstname{Brian} % Your first name
\familyname{Rogers} % Your last name

% All information in this block is optional, comment out any lines you don't need
\title{Curriculum Vitae}
\address{PO BOX 2443}{Valley Center, CA 92082}
\mobile{(760)-207-6247}
%\phone{(000) 111 1112}
%\fax{(000) 111 1113}
\email{BrianJMRogers@gmail.com}
\homepage{BrianJMRogers.github.io}{BrianJMRogers.github.io} % The first argument is the url for the clickable link, the second argument is the url displayed in the template - this allows special characters to be displayed such as the tilde in this example
%\extrainfo{additional information}
%\photo[70pt][0.4pt]{pictures/BriGuy} % The first bracket is the picture height, the second is the thickness of the frame around the picture (0pt for no frame)
%\quote{"A witty and playful quotation" - John Smith}

%----------------------------------------------------------------------------------------

\begin{document}

\makecvtitle % Print the CV title

%----------------------------------------------------------------------------------------
%	EDUCATION SECTION
%----------------------------------------------------------------------------------------
\section{Education}

\cvitemwithcomment{Allegheny College, Meadville PA}{}{2014-2018}
{\textit{Bachelor of Science: }{Computer Science - Software Development}}

{\textit{Minor: }{Economics}

{\textit{GPA: }{3.8}
%----------------------------------------------------------------------------------------
%	SUMMARY OF COMPUTING SKILLS SECTION
%----------------------------------------------------------------------------------------

\section{Computer skills}

\subsection{Languages}
{\textit{Basic Knowledge: }{\textbf{LaTeX, Swift}}}

{\textit{Intermediate Knowledge: }{\textbf{Java, Processing}}}

\subsection{Software}
{\textbf{Git, Vim, Atom, Ubuntu, OS X,}}

%----------------------------------------------------------------------------------------
%	COMPUTER SCIENCE COURSES SECTION
%----------------------------------------------------------------------------------------
\section{Computer Science Courses}

\subsection{Taken}
\begin{itemize}
\item {\textit{Introduction to Computer Science I: }{An introduction to the principles of computer science with an emphasis on algorithmic problem solving and the realization of algorithms using a Java.}}
\item{\textit{Visual Computing: }{An introduction to the fundamentals of computer graphics, visualization, and visual computing. Topics covered include concepts of light, color, two and three dimensional representations, data visualization, image processing, image rendering, and animation. Language: Processing}}
\item{\textit{Introduction to Computer Science II: }{A continuation of Introduction to Computer Science I with an emphasis on data structures, data abstraction, algorithm design, the analytical and experimental evaluation of algorithm performance, and object oriented design and implementation techniques using Java.}}
\item{\textit{Distributed Systems: }{An examination of the principles and paradigms associated with the design, implementation, and analysis of distributed systems.}}
\item{\textit{Game Development with Swift: }{ An independent student-designed program which examines Game Development with Xcode and Swift.}}
\end{itemize}


%----------------------------------------------------------------------------------------
%	PUBLISHED WORKS
%----------------------------------------------------------------------------------------
\section{Published Works}
\cvitemwithcomment{Fratty Bird}{Available for iOS on the App Store}{2016}

%----------------------------------------------------------------------------------------
%	PERSONAL EXPERIENCE SECTION
%----------------------------------------------------------------------------------------

\section{Personal Experience}
\cvitemwithcomment{President of Allegheny College Men's Lacrosse Club}{}{2015-Present}
\begin{itemize}
\item{Communicate with other club presidents, referees, and school faculty to schedule games}
\item{Work with the club's executive board to ensure the success of the club}
\item{Work as liaison between club and all other parties (faculty, Allegheny Student Government, etc.) }
\end{itemize}

\cvitemwithcomment{New Member Educator, Delta Tau Delta Fraternity}{}{Allegheny College, 2016}
\begin{itemize}
\item{Create comprehensive 8-week plan to assimilate new members into the fraternity}
\item{Organize and execute team based activities for new members}
\item{Teach Delta Tau Delta history and traditions to new members}
\item{Draft and present weekly reports concerning the week's events}
\end{itemize}

%----------------------------------------------------------------------------------------

\end{document}